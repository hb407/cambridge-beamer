\documentclass{beamer}

\usetheme{cambridge}

% Standard packages

\usepackage[english]{babel}
\usepackage[version=4]{mhchem}
%\usepackage{times}
%\usepackage[T1]{fontenc}

%\usepackage{fontspec}
%\setsansfont{Myriad Pro}
%\usefonttheme{professionalfonts}

% Setup TikZ

\usepackage{tikz}
\usetikzlibrary{arrows}

%define a few extra symbols
\tikzstyle{block}=[draw opacity=0.7,line width=1.4cm]
\def \hatH {{\skew4\hat{H}}}
\def \hatA {{\skew6\hat{\cal A}}}
\newcommand{\ah}[2] {\skew2\hat{a}_{\vphantom{b}#1}^{#2}}
\newcommand{\ab}[1] {\skew2\hat{a}_{\vphantom{b}\bf #1}}
\def\hatT{\skew3\hat{T}}
\def\hatC {\skew4\hat{C}}
\def \degrees {$^\circ$}
% These are for non-serif fonts
%\def \hatH {{\skew5\hat{H}}}
%\def \hatA {{\skew5\hat{\cal A}}}
%\newcommand{\ah}[2] {\skew3\hat{a}_{#1}^{#2}}
%\newcommand{\ab}[1] {\skew3\hat{a}_{\bf #1}}
%\def\hatT{\skew4\hat{T}}
\newcommand{\dbd}[2] {{\frac{\partial #1}{\partial #2}}}
\newcommand{\braket}[3] {{\langle #1 | #2 | #3 \rangle}}
\newcommand{\brakett}[3] {{\{ #1 | #2 | #3 \}}}
\newcommand{\brket}[2] {{\langle #1 | #2  \rangle}}
\newcommand{\brkett}[2] {{\{ #1 | #2  \}}}
\newcommand{\bket}[1] {{\langle #1  \rangle}}
\newcommand{\D}[1] {D_{\bf #1}}
\newcommand{\ket}[1] {{| #1 \rangle}}
\newcommand{\kD}[1] {\ket{\D{#1}}}
\def\kDz{\ket{D_0}}
\def\bfr{{\bf r}}
\newcommand{\Or}[1] {${\cal O}$[#1]}
\def\PsiCC{\Psi_{\rm CC}}
\def\PsiCI{\Psi_{\rm CI}}
\newcommand{\up}[2] {{^{#1}\!#2}}


% Author, Title, etc.

\title[Holomorphic Hartree--Fock Theory]
{%
  Holomorphic Hartree--Fock Theory
}
\subtitle{The Thom Group: Theory RIG}

\author[Burton, Thom]
{
  \hskip-1.7mm
  Hugh~Burton %\and
}

\institute[Burton and others]
{
%  \inst{1}%
  University of Cambridge
%  \and
%  \vskip-2mm
%  \inst{2}%
%  Bar-Ilan University, Ramat-Gan, Israel
}

\date[December 2016]{Monday 19th December 2016}

% The main document

\begin{document}

\begin{frame}
  \titlepage
\end{frame}

%\begin{frame}{Outline}
%  \tableofcontents
%\end{frame}

\section{Introduction}

\subsection{Self--Consistent Field Method}

\begin{frame}{Self--Consistent Field Method}
 \begin{itemize}
  \item<1-> Electrons in molecule described by single Slater determinant $   \ket{\Psi}$.
  \item<2-> Single Slater determinant $\ket{\Psi}$ is made up of spin orbitals $\chi_i$, $i=1\dots N$.
  $$\ket{\Psi(\mathbf{x}_1, \ldots, \mathbf{x}_N)} =
\frac{1}{\sqrt{N!}}
\left|
   \begin{matrix} \chi_1(\mathbf{x}_1) & \cdots & \chi_N(\mathbf{x}_1) \\
                      \vdots &  \ddots & \vdots \\
                      \chi_1(\mathbf{x}_N) &  \cdots & \chi_N(\mathbf{x}_N)
   \end{matrix} \right|$$
   \item<3-> Spin orbitals are product of a spatial function and one of two spin functions, e.g. $\chi_i(\mathbf{r}) = \phi_i(\mathbf{r}) \ket{\alpha}$.
   \item<4-> Spatial orbitals constructed from chosen basis set 
   $$\phi_i(\mathbf{r})=\sum_\mu^M\eta_\mu(\mathbf{r}) C_{\mu i}.$$
 \end{itemize}
\end{frame} 

\begin{frame}{Self--Consistent Field Method}
 \begin{itemize}
  \item<1-> Take an initial guess for the orbital coefficients $C_{\mu i}$.
  \item<2-> Form density matrix $P_{\mu\nu}=\sum_i^N C_{\mu i} (C_{\nu i})^{*}$.
  \item<3-> Compute Hartree--Fock energy as a functional of density:
  $$E(P_{\mu\nu}) = h_{nuc} + \sum_{\mu\nu}^N P_{\mu\nu} h_{\mu\nu} + \frac{1}{2} \sum_{\mu\nu\sigma\tau}^N P_{\mu\nu} P_{\sigma\tau} \left(2\left(\mu\nu|\sigma\tau\right) - \left(\mu\tau|\sigma\nu\right)\right)$$
  $h_{\mu\nu}=\braket{\chi_\mu}{\hat{h}}{\chi_\nu}$,\ \ $\left(\mu\nu|\sigma\tau\right)=\int\int\frac{\chi_\mu^*(\bfr_1)\chi^*_\sigma(\bfr_2)\chi_\nu(\bfr_1)\chi_\tau(\bfr_2)}{|\bfr_1-\bfr_2|}d^3\bfr_1d^3\bfr_2$
  \vspace{0.5em}
  \item<4-> Solve for $\dbd{E}{P_{\mu\nu}}=0$ (keeping number of electrons fixed).
 \end{itemize}
\end{frame}

\begin{frame}{SCF Solutions to \ce{H2} STO-3G}
Restricted Hartree--Fock:
  \begin{center}
    \includegraphics[scale=0.3]{H2_sto-3g_RHF}
  \end{center}
\end{frame}

\begin{frame}{SCF Solutions to \ce{H2} STO-3G}
Unrestricted Hartree--Fock:
  \begin{center}
    \includegraphics[scale=0.3]{H2_sto-3g_UHF}
  \end{center}
\end{frame}

\subsection{Non--orthogonal Configuration Interaction}
\begin{frame}{Non--orthogonal Configuration Interaction}
 \uncover<1->{SCF states are size--extensive and could be used as a basis for CI methods to produce size--extensive energies.}
 \begin{itemize}
  \item<2-> Different SCF solutions ($\up{x}{\Psi}$ and $\up{y}{\Psi}$) are \alert{not orthogonal}.
  \item<3-> Solve the generalized eigenvalue problem ${\bf H v}=E{\bf S v}$.
  \item<4-> Hamilton matrix elements are given by $H_{xy}=\braket{\up{x}\Psi}{\hat H}{\up y\Psi}$.
  \item<5-> Overlap matrix elements are given by $S_{xy}=\brket{\up{x}\Psi}{\up y\Psi}$.
  \item<6-> Scales as $\mathcal{O}\left( n_s^2\ \mathrm{max}\left(n^3, N^2\right) \right).$
 \end{itemize}
\uncover<3->{\vspace{2em} \tiny A. J. W. Thom and M. Head-Gordon, {\it J. Chem. Phys.} {\bf 131} 124113-1--5, (2009)}
\end{frame}

\begin{frame}{NOCI for \ce{H2} STO--3G}
  \begin{center} 
    \uncover<1-> BUT the number of SCF solutions is \alert{not constant}!!!
  \end{center}
  \begin{center}
    \uncover<2-> {
        \includegraphics[scale=0.38]{H2_normal}
        }
  \end{center}
\end{frame}


\subsection{SCF Solutions to \ce{H2} STO-3G}
\begin{frame}{SCF Solutions to \ce{H2}}
 \vspace{-1em}
 \begin{center}

 \end{center}
\end{frame}


\begin{frame}{SCF Solutions to \ce{H2}}
 \vspace{-1em}
 \begin{center}

 \end{center}

\end{frame}



\section{Holomorphic Hartree--Fock}
\begin{frame}{Holomorphic Hartree--Fock Theory}
 \uncover<1->{Fundamental Theorem of Algebra:}
 \uncover<1->{
  \begin{quote}
   ``Every non-zero, single-variable, \alert{degree $n$ polynomial} with complex coefficients has, counted with multiplicity, exactly \alert{$n$ roots}.''
  \end{quote}}
 \uncover<1->{Can we apply this to $\frac{dE}{d\bf P}=0$?}
 \begin{itemize}
  \item<2->{Must be single-variable polynomial $z$}
  \item<3->{Must have no dependence on $z^*$, but $E=\frac{\braket{\Psi}{\hat H}{\Psi}}{\brket{\Psi}{\Psi}}$ contains $\Psi^*$.}
 \end{itemize}
 \uncover<4->{What happens if we remove dependence of $E$ on $\Psi^*$?} 
\end{frame}

\begin{frame}{Holomorphic energy functional}
 \uncover<1->{Redefine our energy functional as:
 $$\tilde{E}=\frac{\braket{\Psi^*}{\hat H}{\Psi}}{\brket{\Psi^*}{\Psi}}$$}
\uncover<2->{ Now a holomorphic function of several complex variables $C_{\mu i}$}.
 \begin{itemize}
  \item<3-> $\tilde{E}$ is now in general complex.
  \item<4-> Orbitals must be complex--normalised $\sum_{\mu} C_{\mu \cdot} C_{\mu \cdot} = 1.$
  \item<5-> Density matrix now complex--symmetric $\tilde{P_\mu \nu} = \sum_i^N C_{\mu i} C_{\nu i}$.
 \end{itemize}
 \vfill
 \vspace{2em}
 \uncover<1->{\tiny H. G. Hiscock and A. J. W. Thom, {\it J. Chem. Theory Comput.} {\bf 10} 4795--4800,(2014)}
\end{frame}

\begin{frame}{A Geometric Approach}
\begin{itemize}
   \item<1-> Begin with set of N orthonormal orbitals ${ \phi_i }$,
  $$\phi_i(\mathbf{r})=\sum_\mu^M\eta_\mu(\mathbf{r}) C_{\mu i}.$$
   \item<2-> Construct new set of orthonormal orbitals which minimise energy:
  $$\psi_i(\mathbf{r})=\sum_\mu^M\eta_\mu(\mathbf{r}) D_{\mu i}.$$
   \item<3-> Can write optimal set as unitary transformation of initial set:
   $$\mathbf{D} = \mathbf{C} \mathbf{U}.$$
  \end{itemize}
 
\end{frame}

\begin{frame}{A simple case study: \ce{HZ}\ \ STO--3G}
\begin{itemize}
 \item<1->{Two electrons in two basis functions (e.g. \ce{H2} - STO-3G).\\ }
 \item<2->{Points on \textit{standard} Grassmannian represented by one parameter:
 $$ \mathbf{C}(z) = \left(
  \begin{matrix}
  \cos z \\  \sin z
  \end{matrix} \right). $$\\ }
 \item<3->{Extend to $\mathbb{C}-$Orthogonal Grassmannian by taking $z = \theta + i \phi$:\\ 
  $$ \mathbf{C}(\theta, \phi) = \left(
  \begin{matrix}
  \cos \theta \cosh \phi - i \sin \theta \sinh \phi \\  
  \sin \theta \cosh \phi  + i \cos \theta \sinh \phi
  \end{matrix} \right). $$}
 \end{itemize}
\end{frame}

\begin{frame}{A simple case study: \ce{HZ}\ \ STO--3G}


\end{frame}

\begin{frame}{\ce{H2} STO-3G*}
 \vspace{-1em}
 \begin{center}

 \end{center}
 \uncover<1->{\tiny H. G. Hiscock and A. J. W. Thom, {\it J. Chem. Theory Comput.} {\bf 10} 4795--4800, (2014)}
\end{frame} 

\begin{frame}{\ce{H2} 6-31G*}
 \vspace{-1em}
 \begin{center}

 \end{center}
 \uncover<1->{\tiny H. G. A. Burton and A. J. W. Thom, {\it J. Chem. Theory Comput.} {\bf 12} 167--173, (2016)}
\end{frame} 

\section{Future Directions}
\begin{frame}{Future Directions}
 \begin{itemize}
  \item<1->{Currently working to apply geometric optimisation methods on holomorphic energy surface.}
  \item<2->{Hope to understand holomorphic states and the topology of the energy functional.}
  \item<3->{Will investigate the applicability of using NOCI for studying larger molecules.}
  \item<4->{Consider using holomorphic states for other post-HF correlation methods.}
 \end{itemize}
\end{frame}

\begin{frame}{Acknowledgements}
 I would like to thank:
 
 \begin{itemize}
  \item{Alex Thom}
  \item{The Thom Group}
  \item{The Cambridge Trust}
 \end{itemize}
 \begin{center}
  \includegraphics[scale=0.7]{template/CT_logo.jpg}
 \end{center}
\end{frame}
\end{document}
