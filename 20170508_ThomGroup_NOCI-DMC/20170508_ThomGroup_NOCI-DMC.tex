\documentclass{beamer}

\usetheme{cambridge}

% Standard packages

\usepackage[english]{babel}
\usepackage[version=4]{mhchem}

%\usepackage{times}
%\usepackage[T1]{fontenc}

%\usepackage{fontspec}
%\setsansfont{Myriad Pro}
%\usefonttheme{professionalfonts}

% Setup TikZ

\usepackage{tikz}
\usetikzlibrary{arrows}
\usepackage{multimedia}

%define a few extra symbols
\tikzstyle{block}=[draw opacity=0.7,line width=1.4cm]
\def \hatH {{\skew4\hat{H}}}
\def \hatA {{\skew6\hat{\cal A}}}
\newcommand{\ah}[2] {\skew2\hat{a}_{\vphantom{b}#1}^{#2}}
\newcommand{\ab}[1] {\skew2\hat{a}_{\vphantom{b}\bf #1}}
\def\hatT{\skew3\hat{T}}
\def\hatC {\skew4\hat{C}}
\def \degrees {$^\circ$}
% These are for non-serif fonts
%\def \hatH {{\skew5\hat{H}}}
%\def \hatA {{\skew5\hat{\cal A}}}
%\newcommand{\ah}[2] {\skew3\hat{a}_{#1}^{#2}}
%\newcommand{\ab}[1] {\skew3\hat{a}_{\bf #1}}
%\def\hatT{\skew4\hat{T}}
\newcommand{\dbd}[2] {{\frac{\partial #1}{\partial #2}}}
\newcommand{\braket}[3] {{\langle #1 | #2 | #3 \rangle}}
\newcommand{\brakett}[3] {{\{ #1 | #2 | #3 \}}}
\newcommand{\brket}[2] {{\langle #1 | #2  \rangle}}
\newcommand{\brkett}[2] {{\{ #1 | #2  \}}}
\newcommand{\bket}[1] {{\langle #1  \rangle}}
\newcommand{\D}[1] {D_{\bf #1}}
\newcommand{\ket}[1] {{| #1 \rangle}}
\newcommand{\kD}[1] {\ket{\D{#1}}}
\def\kDz{\ket{D_0}}
\def\bfr{{\bf r}}
\newcommand{\Or}[1] {${\cal O}$[#1]}
\def\PsiCC{\Psi_{\rm CC}}
\def\PsiCI{\Psi_{\rm CI}}
\newcommand{\up}[2] {{^{#1}\!#2}}
\def\hE{\tilde{E}}
\newcommand{\II}{\mathrm{i}}

% Author, Title, etc.

\title[MainTitle]
{%
  NOCI Trial Wavefunctions for Diffusion Monte Carlo
}
\subtitle{$\because$ longer acronyms $\Rightarrow$ better methods}

\author
{
  \hskip-1.7mm
  Hugh~Burton %\and
}

\institute[Burton and others]
{
%  \inst{1}%
  University of Cambridge
%  \and
%  \vskip-2mm
%  \inst{2}%
%  Bar-Ilan University, Ramat-Gan, Israel
}

\date{Monday 8th May 2017}

% The main document

\begin{document}

\begin{frame}
  \titlepage
\end{frame}

%\begin{frame}{Outline}
% \tableofcontents
%\end{frame}

\section{Review of DMC}

\begin{frame}{Review of Diffusion Monte Carlo}
  Time-dependent Schr\"{o}dinger equation (TDSE) is given by
  \small{\begin{align}
    \label{TDSE}
    \II \hbar \frac{\partial}{\partial t} \Psi (\mathbf{r}, t) = \hat{H} \Psi (\mathbf{r}, t).
  \end{align}}%
  We recast TDSE in imaginary time, $\tau = \II t$, and apply an energy offset $E_T$
  \small{\begin{align}
  \label{ITDSE}
    - \hbar \frac{\partial}{\partial \tau} \Psi (\mathbf{r}, \tau) = (\hat{H} - E_T) \Psi (\mathbf{r}, \tau).
  \end{align}}%
  In integral form this can be expressed as
  \small{\begin{align}
    \label{integralITDSE}
    \Psi(\mathbf{r},\tau+\delta \tau) =  \sum_k \Psi_k(\mathbf{r}) \mathrm{e}^{-\delta \tau(E_k - E_T)} \langle \Psi_k | \Psi_{init} \rangle.
  \end{align}}%
 In the long time limit, Equation \ref{integralITDSE} projects out lowest eigenstate.
 By adjusting $E_T$, a stable simulation can be reached where $E_T = E_0$. \\
 \vspace{1em}
 \tiny{W. M. C. Foulkes, L. Mitas, R. J. Needs and G. Rajagopal, \textit{Rev. Mod. Phys}, \textbf{73}, 33--83, (2001)
 }%
\end{frame}

\begin{frame}{Review of Diffusion Monte Carlo}
  Wavefunction is represented by a series of walkers that are propagated through imaginary time with importance sampling using a guide wavefunction  $\Psi_T(\mathbf{r})$.
  \begin{enumerate}
    \item{\textbf{Diffusion:} 
      \begin{itemize}
      \item Each electron is propagated to a new position $\mathbf{r} = \mathbf{r}' + \chi + \tau \mathbf{v}_D(\mathbf{r}')$ where $\mathbf{v}_D(\mathbf{r}')$ is the drift velocity 
    \small{\begin{align}
      \label{DriftVelocity}
      \mathbf{v}_D(\mathbf{r}') = \Psi_T(\mathbf{r}')^{-1} \nabla \Psi_T(\mathbf{r}').
    \end{align}}%
    \item Step is accepted using fixed node approximation and with probability 
     \small{\begin{align}
     \label{Paccept}
        \exp \left[ \frac{-\left| \mathbf{r}'-\mathbf{r} - \delta \tau \mathbf{v}_D (\mathbf{r}) \right| ^2}{2 \delta \tau} \right] \left| \frac{\Psi_T(\mathbf{r})}{\Psi_T(\mathbf{r}')} \right| ^2.
     \end{align}}%
     \end{itemize}}%
  \end{enumerate}
\end{frame}

\begin{frame}{Review of Diffusion Monte Carlo}
  \begin{enumerate}
  \setcounter{enumi}{1}
    \item \textbf{Reweighting:} 
      \begin{itemize}
      \item For each diffusion step, walker weight is multiplied by
    \small{\begin{align}
    \label{ReweightWalker}
 \exp \left[ -\delta \tau \left( \frac{ E_{loc}(\mathbf{r}) + E_{loc}(\mathbf{r}') }{2} - E_T \right) \right], 
    \end{align}}%
    where $E_{loc}$ is the local energy given by $ E_{loc}(\mathbf{r}) = \Psi_T (\mathbf{r})^{-1} \hat H \Psi_T (\mathbf{r})$.%
     \end{itemize}
     \item{\textbf{Branching:}
     \begin{itemize}
       \item Walkers with low weights below a certain threshold are killed, whilst those above a certain threshold are copied.
     \end{itemize}
     }
  \item \textbf{Sampling:} 
   \begin{itemize}
       \item Energy is sampled from weighted sum of individual walker local energies
       \small{\begin{align}
       \label{LocalEnergySampling}
 E_{D} \approx \frac{1}{M} \sum_M E_L (\mathbf{r}_M).
    \end{align}}%
     \end{itemize}
  \end{enumerate}
\end{frame}

\begin{frame}{So what's the problem?}
  \begin{itemize}
    \item Quality of trial wavefunction controls statistical efficiency and limits final accuracy of simulation.
    \item Fixed node approximation yields lowest energy for a given nodal surface.
    \item \textbf{Only} when the nodal surface is exactly correct can this give the exact answer.
  \end{itemize}
\end{frame}
\begin{frame}{Could NOCI provide improved trial wavefunctions?}
  \vspace{-0.8em}
  \begin{center}
    \includegraphics[scale=0.45]{BFGS_h-RHF_HLi_STO-3G}
  \end{center}

\end{frame}
\begin{frame}{Could NOCI provide improved trial wavefunctions?}
  NOCI states are constructed from a linear combination of determinants containing holomorphic HF solutions.
\vspace{1em}  
  \begin{itemize}
   \item Wavefunction is constructed from multiple determinants, each based on a different set of molecular orbitals.
   \item Wavefunction is complex valued in general.
  \end{itemize}
  \vspace{1em}  
  Exploiting NOCI trial wavefunctions requires a \alert{multi-determinantal} method and the \alert{fixed phase approximation}.
\end{frame}

\section{Multi-Determinantal DMC}
\begin{frame}{Using a multi-determinantal trial wavefunction}
  Trial wavefunction is given by
    \small{\begin{align}
    \Psi_T = \sum_k^{n_{dets}} a_k \mathrm{det} M^{\uparrow k} \mathrm{det} M^{\downarrow k}.
  \end{align}}
 \hspace{-0.27em}For most of the DMC algorithm we just loop over each reference determinant, but we must be careful when computing $\nabla \Psi_T$ and $\nabla^2 \Psi_T$...
  \small{\begin{align}
     \frac{\nabla_{\mathbf{r}_i} \Psi_T}{\Psi_T} = \frac{ \sum_k^{n_{dets}} a_k \left[ \nabla_{\mathbf{r}_i}  \mathrm{det} M^{\uparrow k} \right] \mathrm{det} M^{\downarrow k}}{ \sum_k^{n_{dets}} a_k \mathrm{det} M^{\uparrow k} \mathrm{det} M^{\downarrow k}}
  \end{align}}
  In the standard case ($n_{dets} = 1$) this reduces to 
  \small{\begin{align}
     \frac{\nabla_{\mathbf{r}_i} \Psi_T}{\Psi_T} = \frac{\nabla_{\mathbf{r}_i}   \mathrm{det} M^{\uparrow}}{  \mathrm{det} M^{\uparrow}} = \sum_{\mu} \left[ \nabla_{\mathbf{r}_i} M^{\uparrow}_{i \mu}(\mathbf{r}) \right]  M^{\uparrow}(\mathbf{r})^{-1}_{\mu i}.
  \end{align}}

  
\end{frame}


\section{Fixed Phase DMC}
  \begin{frame}{Introducing the fixed-phase approximation}
    \begin{itemize}
     \item Take wavefunction as a combination of amplitude and phase
     \small{
      \begin{align}
\Psi \left(\mathbf{r} , \tau \right) = \rho \left( \mathbf{r} , \tau \right) \exp \left[ \II \Phi \left( \mathbf{r} , \tau \right) \right]. 
      \end{align}
      }
      \item Substituting into imaginary time Schr\"{o}dinger equation gives two coupled differential equations
      \small{
      \begin{align}
        \label{FPDMCr}
        - \frac{\partial \rho \left(\mathbf{r} , \tau \right)}{\partial \tau} &= \left[ -\frac{1}{2} \nabla^2 + V(\mathbf{r}) + \frac{1}{2} | \nabla \Phi \left(\mathbf{r} , \tau \right) |^2 \right] \rho \left(\mathbf{r} , \tau \right) \\
        \label{FPDMCi}
        - \frac{\partial \Phi \left(\mathbf{r} , \tau \right)}{\partial \tau} &= \left[ -\frac{1}{2} \nabla^2 + V(\mathbf{r})  + \frac{\nabla \rho \left(\mathbf{r} , \tau \right) \cdot \nabla }{\rho \left(\mathbf{r} , \tau \right)}  \right] \Phi \left(\mathbf{r} , \tau \right).
      \end{align}
      }
      \item Fixed phase approximation sets $\frac{\partial \Phi \left(\mathbf{r} , \tau \right)}{\partial \tau} = 0$, giving $\Phi(\mathbf{r},\tau) = \Phi_{T}(\mathbf{r})$.
     % \item Amplitude $\rho(\mathbf{r},\tau)$ is positive everywhere so method reduces to solving bosonic problem of Equation \ref{FPDMCr}.
    \end{itemize}
     \vspace{1em}
     \tiny{G. Oritz, D. M Ceperley and R. M. Martin, \textit{Phys. Rev. Lett.}, \textbf{71}, 2777, (1993)\\
     C. A. Melton, M. C. Bennett and L. Mitas, \textit{J. Chem. Phys.}, \textbf{144}, 244113, (2016)
 }%
  \end{frame}
  
  \begin{frame}{Implementing FPDMC}
    FPDMC proceeds in same manner as FNDMC with a few modifications:
    \begin{itemize}
      \item{ Drift velocity calculated from trial amplitude $\rho_T(\mathbf{r})$
                \small{\begin{align}
                  \label{DriftVelocityFPDMC}
                  \mathbf{v}_{D}(\mathbf{r}) = \rho_T^{-1} \mathbf{\nabla} \rho_T(\mathbf{r}).
                \end{align}
               }}
      \item{Local energy includes phase dependent repulsive term
                \small{\begin{align}
                  \label{LocalEnergyFPDMC}
                E_{L}(\mathbf{r}) =  \rho_T^{-1}(\mathbf{r}) \left[ -\frac{1}{2} \nabla^2 + V(\mathbf{r})  + \frac{1}{2} | \mathbf{\nabla} \Phi_T \left(\mathbf{r} \right) |^2 \right] \rho_T \left(\mathbf{r}  \right).
               \end{align}
               }}
     \end{itemize}
     Straightforward expressions for Equations \ref{DriftVelocityFPDMC} and \ref{LocalEnergyFPDMC} can be derived from $\nabla \Psi_T$ and $\nabla^2 \Psi_T$
     \begin{center}
               \vspace{-1.em}
               \small{\begin{align}
                  \rho_T^{-1}(\mathbf{r}) \mathbf{\nabla} \rho_T(\mathbf{r}) &= \Re\left( \Psi_T^* \mathbf{\nabla} \Psi_T \right) / \rho_T^2, \\
                   E_{L}(\mathbf{r}) &= - \frac{1}{2}\Re\left( \Psi_T^* \nabla^2 \Psi_T \right) / \rho_T^2 + V(\mathbf{r}).
                \end{align}}
     \end{center}
  \end{frame}
  
  \begin{frame}{Fixed-node as a special case of fixed-phase}
  Construct a complex trial function from real and imaginary part
  \begin{align}
    \Psi_T = \Psi_{R} + \mathrm{i} \epsilon \Psi_{B}
  \end{align}
  Potential generate by phase of $\Phi_T (\mathbf{r})$ is 
  \begin{align}
    V_{ph} = \frac{1}{2} \left| \frac{\epsilon \left( \Psi_R \nabla \Psi_B - \Psi_B \nabla \Psi_R \right)}{\Psi_R^2 + \epsilon^2 \Psi_B^2} \right|^2.
  \end{align}
  \begin{itemize}
  \item Away from nodal surface ($\Psi_R \neq 0$), limit $\epsilon \rightarrow 0$ yields $V_{ph} =0$.
  \item At the nodal surface ($\Psi_R = 0$), generally $\left| \nabla \Psi_R \right|^2 \geq 0$ so limit $\epsilon \rightarrow 0$ yields $V_{ph} = \infty$.
  \end{itemize}
\end{frame}

\begin{frame}{Putting it all together}
  Combining multi-determinantal DMC and FPDMC should require minimal alterations to the DMC algorithm.\\
  \begin{itemize}
    \item Implement multi-determinantal algorithm and test for real wavefunctions.
    \item Back compatibility achieved by setting $n_{det} = 1$.
    \item Independent of Jastrow factor implementation.
    \item Code up the FPDMC and test using NOCI states constructed from holomorphic solutions.
  \end{itemize}
  Method is being written into the group \texttt{simpleDMC} code.
\end{frame}

\begin{frame}{Questions to investigate}
  Beyond implementation, the interesting technical questions include
  \begin{itemize}
    \item Do NOCI trial functions give more efficient convergence?
    \item If an excited NOCI state is used, do we converge onto the related FCI state?
    \item What result would we obtain for `dormant' holomorphic solutions that are complex everywhere?
  \end{itemize}
  Using the new framework we will hopefully be able to use NOCI trial wavefunctions for wider applications!
\end{frame}



%\begin{frame}{Acknowledgements}
% I would like to thank:
% \begin{itemize}
%  \item{Alex Thom}
% \item{The Thom Group}
% \item{The Cambridge Trust}
% \end{itemize}
% \begin{center}
%  \includegraphics[scale=0.7]{../template/CT_logo.jpg}
% \end{center}
%\end{frame}

\end{document}
