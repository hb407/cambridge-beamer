\documentclass{beamer}

\usetheme{cambridge}

% Standard packages

\usepackage[english]{babel}
\usepackage[version=4]{mhchem}
%\usepackage{times}
%\usepackage[T1]{fontenc}

%\usepackage{fontspec}
%\setsansfont{Myriad Pro}
%\usefonttheme{professionalfonts}

% Setup TikZ

\usepackage{tikz}
\usetikzlibrary{arrows}

%define a few extra symbols
\tikzstyle{block}=[draw opacity=0.7,line width=1.4cm]
\def \hatH {{\skew4\hat{H}}}
\def \hatA {{\skew6\hat{\cal A}}}
\newcommand{\ah}[2] {\skew2\hat{a}_{\vphantom{b}#1}^{#2}}
\newcommand{\ab}[1] {\skew2\hat{a}_{\vphantom{b}\bf #1}}
\def\hatT{\skew3\hat{T}}
\def\hatC {\skew4\hat{C}}
\def \degrees {$^\circ$}
% These are for non-serif fonts
%\def \hatH {{\skew5\hat{H}}}
%\def \hatA {{\skew5\hat{\cal A}}}
%\newcommand{\ah}[2] {\skew3\hat{a}_{#1}^{#2}}
%\newcommand{\ab}[1] {\skew3\hat{a}_{\bf #1}}
%\def\hatT{\skew4\hat{T}}
\newcommand{\dbd}[2] {{\frac{\partial #1}{\partial #2}}}
\newcommand{\braket}[3] {{\langle #1 | #2 | #3 \rangle}}
\newcommand{\brakett}[3] {{\{ #1 | #2 | #3 \}}}
\newcommand{\brket}[2] {{\langle #1 | #2  \rangle}}
\newcommand{\brkett}[2] {{\{ #1 | #2  \}}}
\newcommand{\bket}[1] {{\langle #1  \rangle}}
\newcommand{\D}[1] {D_{\bf #1}}
\newcommand{\ket}[1] {{| #1 \rangle}}
\newcommand{\kD}[1] {\ket{\D{#1}}}
\def\kDz{\ket{D_0}}
\def\bfr{{\bf r}}
\newcommand{\Or}[1] {${\cal O}$[#1]}
\def\PsiCC{\Psi_{\rm CC}}
\def\PsiCI{\Psi_{\rm CI}}
\newcommand{\up}[2] {{^{#1}\!#2}}


% Author, Title, etc.

\title[Holomorphic Hartree--Fock Theory]
{%
  Holomorphic Hartree--Fock Theory
}
\subtitle{The Thom Group: Theory RIG}

\author[Burton, Thom]
{
  \hskip-1.7mm
  Hugh~Burton %\and
}

\institute[Burton and others]
{
%  \inst{1}%
  University of Cambridge
%  \and
%  \vskip-2mm
%  \inst{2}%
%  Bar-Ilan University, Ramat-Gan, Israel
}

\date[P2P 2016]
{Peer-to-Peer Presentations, Michaelmas 2016}

% The main document

\begin{document}

\begin{frame}
  \titlepage
\end{frame}

\section{Introduction}

\subsection{Self--Consistent Field Method}
\begin{frame}{The Big Picture}
\begin{columns}
 \begin{column}{0.55\textwidth}
 \uncover<1-> {How might we model biomolecular interactions?}
  \vspace{1em}
  \begin{enumerate}
   \item<2-> Simulate using molecular dynamics.
   \item<3-> Molecular dynamics requires a force field.
   \item<4-> Find parameters using \textit{ab initio} methods.
  \end{enumerate}
 \end{column}
 \begin{column}{0.45\textwidth}
 \includegraphics<-1>[scale=0.375]{motivations1.png}
 \includegraphics<1-2>[scale=0.375]{motivations2.png}
 \includegraphics<2-3>[scale=0.375]{motivations3.png}
 \includegraphics<3-4>[scale=0.375]{motivations4.png}
 
 \end{column}
\end{columns}
 \vspace{-1.5em}
 \uncover<1->{\tiny http://www.man.poznan.pl/CBB/research.html\\}
 \uncover<2->{\tiny D. J. Wales and T. Head--Gordon, {\it J. Phys. Chem. B}, {\bf 116}, 8394--8411, (2012)}
\end{frame}

\begin{frame}{Hartree--Fock Theory}
 \begin{itemize}
  \item<1-> Electrons in molecule described by a wave function $\ket{\Psi}$.
  \item<2->{Wave function satisfies the Schr\"{o}dinger equation $$\hat H \ket{\Psi} = E \ket{\Psi}$$}
  \item<3-> Approximate electron--electron repulsion as spherical average.
   \end{itemize}
    \uncover<3->{
    \begin{center}
     \includegraphics<1->[scale=0.2]{hartreeFock}
    \end{center}}
   
\end{frame}

\begin{frame}{Self--Consistent Field Method}
 \uncover<1->{How can we compute the wave function?}
 \begin{enumerate}
  \item<2-> Make an \alert{initial guess} of wave function $\ket{\Psi_{0}}$.
  \item<3->  Calculate charge density to generate potential $V(\mathbf{r})$.
  \item<4->  Solve the Schr\"{o}dinger equation to compute new wavefunction $\ket{\Psi}$.
  \item<5-> Repeat 2. and 3. until \alert{no further change} to $\ket{\Psi}$.
  \end{enumerate}
   \vspace{1em}
  \uncover<6->{The energy is then computed by the expectation value
  $$\left< E \right> = \frac{\braket{\Psi}{\hat H}{\Psi}}{\brket{\Psi}{\Psi}}.$$
  Final $\ket{\Psi}$ is now a \alert{stationary--point} of $\left< E \right>$.}
\end{frame}

\subsection{SCF Solutions to \ce{H2} STO-3G}
\begin{frame}{SCF Solutions to \ce{H2}}
 \uncover<1->{
  Restricted Hartree--Fock (RHF): \\
  \hspace{1.5em} Spin--up and spin--down electrons must have same set of spatial orbitals.
 }
 \uncover<2->{
  \begin{center}
   \includegraphics<1->[scale=0.32]{H2_sto-3g_RHF}
  \end{center}
 }
\end{frame}


\begin{frame}{SCF Solutions to \ce{H2}}
 \uncover<1->{
  Unrestricted Hartree--Fock (UHF): \\
  \hspace{1.5em} Spin--up and spin--down electrons can have different spatial orbitals.
 }
 \uncover<2->{
  \begin{center}
   \includegraphics<1->[scale=0.32]{H2_sto-3g_UHF}
  \end{center}
 }
\end{frame}

\section{Configuration Interaction}
\begin{frame}{Introducing Electron Correlation}
 \uncover<1->{Full Configuration Interaction:}
  \begin{itemize}
   \item<2->{Wave function is constructed as a linear combination of excited states from a given reference.}
   \item<3-> {e.g. for $\ket{\sigma_{g}^{2}}$ reference in \ce{H2}:
             $$\ket{\Psi_0^{\mathrm{FCI}}} = C_{1} \ket{\sigma_{g}^{2}} + C_{2} \ket{\sigma_{g}^{1}\sigma_{u}^{1}} + C_{3} \ket{\sigma_{u}^{1}\sigma_{g}^{1}} + C_{4} \ket{\sigma_{u}^{2}}$$}
  \end{itemize}
  \uncover<0-> {\vspace{-1.5em}}
  \begin{center}
   \includegraphics<3->[scale=0.28]{FCIdiag}
  \end{center}
  \begin{itemize}
 \uncover<0-> {\vspace{-1.5em}}
  \item<4-> {Gives \alert{exact energy} but is expensive and scales as $\mathcal{O}(N!)$ or $\mathcal{O}(N^N)$.}
  \end{itemize}
\end{frame}
 
\subsection{Non--orthogonal Configuration Interaction}
\begin{frame}{Non--orthogonal Configuration Interaction}
 \uncover<1->{Can we use the SCF states instead?}
 \begin{itemize}
  \item<2-> RHF best describes equilibrium; UHF best describes dissociation.
  \item<3-> SCF states are not mutually orthogonal, $\brket{\up{x}\Psi}{\up{y}\Psi} \neq \delta_{x y}.$
  \end{itemize}
   \vspace{1em}
 \uncover<4->{We must use Non-orthogonal Configuration Interaction.\\ 
 \begin{itemize}
 \item This scales as $\mathcal{O}\left( n_s^2\ \mathrm{max}\left(n^3, N^2\right) \right)$.}
 \end{itemize} 
 \vspace{1em}
 \begin{center}
 \uncover<5->{But the number of SCF solutions is \alert{not constant} across all geometries!}
 \end{center}
\vspace{2em}
\uncover<4->{\tiny A. J. W. Thom and M. Head-Gordon, {\it J. Chem. Phys.} {\bf 131} 124113-1--5, (2009)}
\end{frame}

\subsection{Holomorphic Hartree--Fock Theory}
\begin{frame}{Holomorphic Hartree--Fock Theory}
 \uncover<1->{Fundamental Theorem of Algebra:
  \begin{quote}
   ``Every non-zero, single-variable, \alert{degree $n$ polynomial} with complex coefficients has, counted with multiplicity, exactly \alert{$n$ roots}.''
  \end{quote}}
  \vspace{-1.5em}
 \begin{itemize}
  \item<2->{Must be single-variable polynomial $z$.}
  \item<3->{Must have no dependence on $z^*$.}
 \end{itemize}
 \uncover<4-> {What happens if we remove dependence of $\left< E \right>$ on $\Psi^*$?
 $$\left< \tilde E \right> = \frac{\braket{\Psi^*}{\hat H}{\Psi}}{\brket{\Psi^*}{\Psi}}.$$}
 \vspace{1em}
 \uncover<4->{\tiny H. G. Hiscock and A. J. W. Thom, {\it J. Chem. Theory Comput.} {\bf 10} 4795--4800, (2014)}
\end{frame}

\begin{frame}{\ce{H2} STO-3G*}
 \vspace{-1em}
 \begin{center}
  \includegraphics<1->[scale=0.4]{HoloHF/H2holo-3}
 \end{center}
 \uncover<1->{\tiny H. G. Hiscock and A. J. W. Thom, {\it J. Chem. Theory Comput.} {\bf 10} 4795--4800, (2014)}
\end{frame} 

\begin{frame}{\ce{H2} 6-31G*}
 \vspace{-1em}
 \begin{center}
  \includegraphics<1->[scale=0.4]{HoloHF/H2_4basis}
 \end{center}
 \uncover<1->{\tiny H. G. A. Burton and A. J. W. Thom, {\it J. Chem. Theory Comput.} {\bf 12} 167--173, (2016)}
\end{frame} 

\section{Future Directions}
\begin{frame}{Future Directions}
 \begin{itemize}
  \item<1->{Currently working to develop methods for reliably finding hUHF states.}
  \item<2->{Hope to understand structure of holomorphic states.}
  \item<3->{Will study applicability of NOCI for larger molecules.}
 \end{itemize}
\end{frame}

\begin{frame}{Acknowledgements}
 I would like to thank:
 
 \begin{itemize}
  \item{Alex Thom}
  \item{The Thom Group}
  \item{The Cambridge Trust}
 \end{itemize}
 \begin{center}
  \includegraphics[scale=0.7]{CT_logo.jpg}
 \end{center}
\end{frame}
\end{document}
